%%%%%%%%%%%%%%%%%%%%%%%%%%%%%%%%%%%%%%%%%%%%%%%%%%%%%%%%%%%%%%%%%%
%%%%%%%% CPSC 66 FALL 2021  REPORT %%%%%%%%%%%%%%%%%%%%%%%%
%%%%%%%% This template is modified from ICML 2014 %%%%%%%%%%%%%%%%
%%%%%%%%%%%%%%%%%%%%%%%%%%%%%%%%%%%%%%%%%%%%%%%%%%%%%%%%%%%%%%%%%%

\documentclass{article}

%include any external packages here.  This is similar to loading a
%library in python or C++

% use Times
\usepackage{times}
% For figures
\usepackage{graphicx}
\usepackage{subfigure}

% For citations
\usepackage{natbib}

% For algorithms and pseudocode
\usepackage{algorithm}
\usepackage{algorithmic}

%Adds hyperlinks to your citations automatically
\usepackage{hyperref}

% Packages hyperref and algorithmic misbehave sometimes.  We can fix
% this with the following command.
\newcommand{\theHalgorithm}{\arabic{algorithm}}

\usepackage[accepted]{icml2014}


% If your title is long (below), use this command to also provide
% short version.  This will go on the top of every page
\icmltitlerunning{Final Report}

\begin{document}

\twocolumn[ %use two column if you need a text to span across the whole page
\icmltitle{ CPSC 66 Final Report: \\ % \\ force a new line
Examples and Requirements }

\icmlauthor{Student 1 Name}{userID1@swarthmore.edu}
\icmlauthor{Student 2 Name}{userID2@swarthmore.edu}

\vskip 0.3in
]

\begin{abstract}
A one- or two-paragraph abstract that outlines the central goal and
  results of the project.  This is your 30-second elevator pitch where you
  sell a reader on reading your paper.  It should be 200 words maximum.
\end{abstract}

\section{Introduction}
\label{introduction}

What you attempted to do and what
  was the motivation for your work. You should provide some context about the problem
  including any relevant background about the task and related work.

\section{Methods}
\label{methods}

A description of your approach to solving the task.  Provide algorithms, equations,
descriptive figures, pseudocde, etc.

\section{Experiments and Results}
\label{results}

Describe your experimental methodology (details about the data, preprocessing,
experimental methodology, and performance measures utilized).

\section{Discussion}
\label{discussion}

Analyze your results.  Note that this is merely a suggested structure; you can
instead interweave results and discussion if you have multiple experiments that
you ran.  You can multiple results sections if there was a distinct set of
tasks you were completing (as in Lab 3 where we separated the train/tune/test
experiments from the learning curves).


\section{Conclusions}
\label{conclusion}

Lessons learned.  Wrap up the paper with a restatement of the initial hypothesis
and your findings.  Discuss unanswered questions/possible future work to further
the study of this central question.

\section*{Acknowledgments}


Place acknowledgements in an unnumbered section at the
end of the paper. Typically, this will include
to colleagues who contributed to the ideas, individuals who
reviewed your submission, or external sources who helped
acquire data.


% In the unusual situation where you want a paper to appear in the
% references without citing it in the main text, use \nocite
\nocite{langley00}

\bibliography{references}
\bibliographystyle{icml2014}

\end{document}


% This document was modified from the file originally made available by
% Pat Langley and Andrea Danyluk for ICML-2K. This version was
% created by Lise Getoor and Tobias Scheffer, it was slightly modified
% from the 2010 version by Thorsten Joachims & Johannes Fuernkranz,
% slightly modified from the 2009 version by Kiri Wagstaff and
% Sam Roweis's 2008 version, which is slightly modified from
% Prasad Tadepalli's 2007 version which is a lightly
% changed version of the previous year's version by Andrew Moore,
% which was in turn edited from those of Kristian Kersting and
% Codrina Lauth. Alex Smola contributed to the algorithmic style files.
